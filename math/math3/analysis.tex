\documentclass[a4paper,11pt,twocolumn]{jsbook}
\usepackage{amsthm} %theorem
\usepackage{amsmath} %cases
\usepackage{amssymb} %\becase
\usepackage{ascmac} %box text
\usepackage{bm}
%\newtheorem{def}{定義}[section]
\newtheorem{thm}{定理}[section]
\newtheorem{example}{例題}[section]
\newtheorem*{solution}{解答}


\begin{document}
\title{解析}
\author{小川 雅弘}
\maketitle


\chapter{極限}

\begin{screen}
  \begin{example}
    $a>1$のとき, $\lim_{x \to \infty} \frac{x^k}{a^x} = 0$ を示せ.
  \end{example}
\end{screen}
\begin{solution}
  $t>0$のとき,$f(t)=a^t$に対し,$f'(t)=a^t \log a, f''(t)=a^t (\log a)^2 > 0 \ (\because a>1)$だから
  $f(t)$は常に接線より上にあり,t=0における接線$y=f'(0)t+f(0) = t \log a + 1$よりも上にあるので,
  \begin{eqnarray*}
    a^t > t\log a + 1 > t\log a \\
    \Rightarrow a^{(k+1)t} > (t \log a)^{k+1}
  \end{eqnarray*}
  $(k+1)t=x$とおけば,
  \begin{eqnarray*}
    0 < \frac{x^k}{a^x} < \frac{1}{x} \left( \frac{k+1}{\log a} \right)^{k+1} 
    \xrightarrow[x \to \infty]{} 0
  \end{eqnarray*}
  よって示せた. \qed
\end{solution}

\section{無限等比級数}

\begin{screen}
  \begin{thm}
    無限等比級数の和 $S = \sum_{n=1}^\infty a r^{n-1}$ は, 
    $r \neq 1$ のとき,
    \begin{equation}
      S_n = \sum_{k=1}^n a r^{k-1} = a \frac{1-r^n}{1-r}
    \end{equation}
    より,
    \begin{equation}
      S =
      \begin{cases}
        0 & a=0\\
        \frac{a}{1-r} & -1<r<1\\
        発散 & otherwise.
      \end{cases}
    \end{equation}
  \end{thm}
\end{screen}

\begin{screen}
  \begin{example}
    以下の循環小数を分数に直せ.
    $0.\dot{1}\dot{2}$
  \end{example}
\end{screen}
\begin{solution}
  \begin{eqnarray}
    0.\dot{1}\dot{2} = \frac{12}{100} + \frac{12}{100^2} + \dots 
    = \frac{12/100}{1-1/100} \nonumber \\
    = \frac{12}{99} = \bm{\frac{4}{33}} \nonumber
  \end{eqnarray}
\end{solution}


\chapter{微分法}

\begin{screen}
  \begin{example}
    円$C$: $x^2+(y-1)^2=1/4$上の点$P$と,曲線D: $y=\log x$上の点Q
    を結ぶ線分PQの長さの最小値を求めよ.
  \end{example}
\end{screen}
\begin{solution}
  曲線D上の任意の点Qに対し,円C上の点PでPQが最短となるのは,
  Pが円Cの中心$A(0,1)$とQを結ぶ線分AQ上にあるときであり,
  このとき最短距離は,
  \begin{eqnarray}
    PQ = AQ - 1/2
    \label{eq:minpq}
  \end{eqnarray}
  である.よってAQの最小値を求めれば良い.

  Q$(x,\log x)\ (x>0)$とおくと,
  \begin{eqnarray}
    f(x) := AQ^2
    = x^2 + (\log x - 1)^2 \nonumber \\
    \Rightarrow f'(x) = 2x + 2(\log x - 1)\frac{1}{x} \nonumber \\
    = \frac{2}{x} (x^2 + \log x - 1) 
    =: \frac{2}{x} g(x) \nonumber \\
    \Rightarrow g'(x) = 1/x + 2x > 0 \ (x>0) \nonumber
  \end{eqnarray}
  より$g(x)$は単調増加関数であり, $g(1) = 0$だから,
  $f(x)$の増減表は下表のようになる.
  \begin{eqnarray}
    \begin{array}{c|cccc}
      x & 0 & \dots & 1 & \dots \\ \hline
      g && - & 0 & + \\
      f' && - & 0 & + \\
      f && \searrow & \min & \nearrow \nonumber
    \end{array}
  \end{eqnarray}

  よって,
  \begin{eqnarray}
    \min AQ = \sqrt{f(1)} = \sqrt{2} \nonumber 
  \end{eqnarray}
  だから(\ref{eq:minpq})より,
  \begin{eqnarray}
    \min PQ = \bm{\sqrt{2} - \frac{1}{2}} \nonumber
  \end{eqnarray}
\end{solution}

\begin{screen}
  \begin{example}
    方程式 $a 2^x - x^2 = 0$ が異なる3つの解をもつような実数aの範囲を求めよ.
  \end{example}
\end{screen}
\begin{solution}
  \begin{eqnarray}
    a 2^x - x^2 = 0 \label{eq:numsolution}\\
    \Rightarrow a = \frac{x^2}{2^x} \nonumber
  \end{eqnarray}
  より,(\ref{eq:numsolution})の実数解は,曲線 $y=f(x):=\frac{x^2}{2^x}$と
  直線$y=a$の交点のx座標に等しいので,これらが異なる3交点を持てば良い.
  そこで$y=f(x)$のグラフを描く.
  \begin{eqnarray*}
    f'(x) = \frac{2x2^x - x^22^x\log 2}{2^{2x}} = \frac{x}{2^x}(2-x\log 2)\\
    f(-\infty)=+\infty, 
    f(+\infty)=0, 
    f(0)=0\\
    f(2/\log 2) = \frac{(2/\log2)^2}{2^{2/\log2}} = \frac{4/(\log2)^2}{2^{2\log e}} 
    = \frac{4}{e^2 (\log2)^2}
  \end{eqnarray*}
  よって増減表は下図のようになる.
  \begin{eqnarray*}
    \begin{array}{c|ccccccc}
      x&-\infty&\dots&0&\dots&\frac{2}{\log2}&\dots&+\infty\\
      f'&&-&0&+&0&-&\\
      f&+\infty&\searrow&0&\nearrow&\frac{4}{e^2(\log2)^2}&\searrow&0
    \end{array}
  \end{eqnarray*}
  ゆえに求めるaの範囲は,
  $\bm{0<a<\frac{4}{e^2(\log2)^2}}$
\end{solution}

\end{document}